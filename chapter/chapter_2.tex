\documentclass[../main.tex]{subfiles}

\begin{document}

\chapter{Feasibility Study}

\section{Phase Imprint}

In order to produce a soliton in a condensate the \textit{phase imprint} technique has been already exploited in several experiments << Insert references>>.\\
The \textit{phase imprint} consist in shining half the condensate with laser light in order to produce an optical dipole potential. The potential produce a change of the wavefunction phase in the illuminated part of the condensate. To produce a soliton the relative phase between the two sides of the condensate must be equal to $\Delta \phi = \pi$.\\
The dependence of the phase to the dipole potential is expressed by the following equation:

\begin{equation}
\Delta \phi = 2 \frac{U \tau}{\hbar}
\end{equation}

where \textbf{U} is the contribute of the dipole potential and $\tau$ is the exposition time to the external potential.\\
Another important factor to take into account is the scattering rate, it should stay $\gamma \ll 1$ otherwise the atoms are heated by the light. 


\section{Stak Shift from Dipole Potential}


\section{Sample Preparation}

\subsection{Dipole Trap}

\subsection{Population Transfer Theory}

\end{document}