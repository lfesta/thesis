\documentclass[../thesis.tex]{subfiles}

\begin{document}

\chapter{Phase Imprint Beam Generation and Alignment}

\section{Laser Source}

The laser source for the phase imprint beam is provided by the D1 setup. The wavelength of the emitted laser light is checked by sapling the beam and sending, though an optical fiber, a part of the beam to a wavemeter. Then the wavelength is regulated through the hex key at the side of the cavity, see Fig. \ref{fig:mastlas}, that controls roughly the position of the grating in the horizontal direction. A finer tuning is performed with the control of the piezoelectric crystal attached to the grating, controlled by the offset of the PID, the current source and the TEC control for the temperature.\\
The the laser light is amplified by the RSA amplifier and the frequency is doubled by the doubling cavity. The laser has been left free running during the experiments controlling time to time the wavelength of the emitted light to check the stability.
The laser has proven to be stable at the GHz with only drifts of about hundred of MHz during hours of work.\\

During the alignment procedure the phase imprint beam has been used as repumper light, locking the laser on the crossover of the transition F = 1$\rightarrow$F$^\prime$ = 2, it permits the circulation of the atoms from the dark state to the imaging beam F = 1 to the bright state F = 2 through spontaneous emission.

\section{Alignment of the Phase Imprint Beam}

The alignment has been performed through the newly installed imaging setup at the bottom of the trap.
Using the camera the laser beam, not shadowed, has been superimposed to the condensate using mirrors and the in-situ position of the condensate as reference. After the alignment on the atoms we placed the knife edge to shadow half of the laser beam image. Then the position of the blade's image has been corrected using the same mirror as before, the image of the blade has been positioned on the centre of the condensate.\\

The blade has been placed roughly on the lens focus' position and the fine tuning has been done with the translation stage. The optimization of the blade's focus has been done in two steps.\\
In the first step, the focus has been optimized looking at the image of the blade from the camera in live mode. The optimization has been performed minimizing the diffraction pattern of the blade seen through the camera.\\
In the second step, the light from the D1 setup has been used as repumper light, as explained before, to make the imaging of half of the insitu condensate. In this way a finer optimization on the diffraction pattern and on the blade's position has been obtained.\\
The probe light used for the imaging is provided superimposing the probe beam with the phase imprint beam using a PBS cube as shown in the schematic of the phase imprint, see Fig. <<Inserire ref figura>>.\\


\end{document}