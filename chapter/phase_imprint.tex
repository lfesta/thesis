\documentclass[../main.tex]{subfiles}

\begin{document}

We want to introduce a difference of phase $\Delta \phi = \pi$, between two parts of the condensate. In this way we create a soliton at the boundary of the two.\\
The phase of the condensate's wavefunction can be modifyed shining it with a ligth far from resonance, in order to minimize the scattering losses. In this way the potential feeled by the atoms is a dipole potential, the phase shift can be computed using the following formula:

\begin{equation}
\Delta \phi = 2 \frac{U \tau}{\hbar}
\end{equation}

where U is the term referring to the potential and $\tau$ the exposition time.\\
The beam is centered at the half of the condensate and since we want a relative shift $\Delta \phi = \pi$ between the two sides of the condensate, the beam is shadowed in order to shine only one side. The shadowing is obtained using a knife edge to cut out half of the beam and a telescope with magnification \textcolor{red}{Inserire ingrandimento}, to imagine the knife edge on the atoms.\\
In the experiment we have a condensate composed by two spin components respectively in the fundamental state, F=1,m$_F$=-1 and F=1,m$_F$=0 also F=1,m$_F$=-1 and F=1,m$_F$=+1. Our aim is to imprint the condensate in a way to produce a relative shift on the same component equal to $\pi$ and a relative phase shift between the two components of $2\pi$. Then a condition on the potential feeled by the two components arises:

\begin{equation}
U_{mf} = -U_{mf^\prime}
\end{equation}

Such condition can be satisfied exploting the scalar part of the dipole potential, that is usually neglected due the far away detuning from the transition frequency. The complete formulation of the potential is the following:

\begin{equation}
U = \frac{\hbar \Gamma I_=0}{24 I_S} \left[ \left( \frac{1}{\delta_{1/2}} + \frac{2}{\delta_{3/2}} \right) - g_F m_F \sqrt{1 - \epsilon^2} \left( \frac{1}{\delta_{1/2}} - \frac{1}{\delta_{3/2}} \right) \right]
\end{equation}

where $\Gamma$ is the natural linewidth, m$_F$ is the Zeeman sublevel of the atoms, g$_F$ is the hyperfine Landé g factor, I$_S$ the saturation intensity $I_S = 2\pi^2 \hbar c \Gamma /(3\lambda^3)$ and I$_0$ the peak intensity $2P/(\pi\textit{w}^2_0)$ in terms of the laser power. The detunings are in units of $\Gamma$ and represents the difference between the laser frequency and the D$_1$ and D$_2$ transition frequencies. The right hand side term represent the optical Zeeman splitting and depends on the polarization of the light through $\epsilon$ term, for circular polarized light $\epsilon = 0$.\\
Imposing the previous condition in the two cases on study, we have found two different frequencies, in the case with m$_F = -1,0$ the frequency is $\omega_l = D_1 + 128.88 GHz$ and in the case with m$_F = -1,+1$ the frequency is $\omega_l = D_1 + 171.84 GHz$.\\

At the output of the fiber we have a power of $P = 200 mW$, the beam is collimated by a lens with focal length of $11 mm$ with a waist of $\textit{w} = 0.992 mm$

\end{document}